\documentclass[twoside]{article}


% ------
% Fonts and typesetting settings
\usepackage[sc]{mathpazo}
\usepackage[T1]{fontenc}
\linespread{1.05} % Palatino needs more space between lines
\usepackage{microtype}


% ------
% Page layout
\usepackage[hmarginratio=1:2,top=32mm,columnsep=20pt]{geometry}
\usepackage[font=it]{caption}
\usepackage{paralist}
\usepackage{multicol}

% ------
% Lettrines
\usepackage{lettrine}


% ------
% Abstract
\usepackage{abstract}
	\renewcommand{\abstractnamefont}{\normalfont\bfseries}
	\renewcommand{\abstracttextfont}{\normalfont\small\itshape}


% ------
% Titling (section/subsection)
\usepackage{titlesec}
\renewcommand\thesection{\Roman{section}}
\titleformat{\section}[block]{\large\scshape\centering}{\thesection.}{1em}{}


% ------
% Header/footer
\usepackage{fancyhdr}
	\pagestyle{fancy}
	\fancyhead{}
	\fancyfoot{}
	\fancyhead[C]{Journal paper template $\bullet$ April 2012 $\bullet$ Vol. XXI, No. 1}
	\fancyfoot[RO,LE]{\thepage}


% ------
% Clickable URLs (optional)
\usepackage{hyperref}

% ------
% Maketitle metadata
\title{\vspace{-15mm}%
	\fontsize{24pt}{10pt}\selectfont
	\textbf{Digital Signal Processing}
	}	
\author{%
	\large
	\textsc{Jonathan S. Doe}\thanks{Template by \href{http://www.howtotex.com}{howtoTeX.com}} \\[2mm]
	\normalsize	University of Technology, Delft \\
	\normalsize	\href{mailto:frits@howtoTeX.com}{frits@howtoTeX.com}
	\vspace{-5mm}
	}
\date{}



%%%%%%%%%%%%%%%%%%%%%%%%
\begin{document}

\maketitle
\thispagestyle{fancy}

\begin{abstract}
\noindent This experiment experiment explored the benifits and parameter space of digital signal processing. A high-purity germanium detector was coupled simultaneously to both a digital and analog data-acquisition system. The parameters were optimized for both configurations. For the analog system, the shaping time was corrected to find a minimum of full width half maximum (FWHM) of a $59.54 KeV$ $^{241}Am$ $\gamma$ line. For the digital system, the discretete convolution developed by jordanov[1] as a trapazoidal shaping filter was interpreted using linear algebra methods, the fano factor was derived, and the shaping time and gap time were optimized using a combination of $\gamma$ lines from  $^{241}Am$ and $^{60}Co$
\end{abstract}
	

\begin{multicols}{2}

\section{Introduction}


\section{Another section}
\begin{compactitem}
\item Donec dolor arcu, rutrum id molestie in, viverra sed diam.
\item Curabitur feugiat, 
\item turpis sed auctor facilisis, 
\item arcu eros accumsan lorem, at posuere mi diam sit amet tortor. 
\item Fusce fermentum, mi sit amet euismod rutrum, 
\item sem lorem molestie diam, iaculis aliquet sapien tortor non nisi. \item Pellentesque bibendum pretium aliquet. 
\end{compactitem}

\section{One more section}
Vestibulum tincidunt accumsan nisl dapibus rhoncus. Phasellus pulvinar suscipit magna ut consectetur. Curabitur ipsum dui, consectetur id dignissim nec, sollicitudin id lacus. Aenean volutpat, neque sit amet luctus porta, neque dui pretium ipsum, non ultricies justo nibh id justo. Cras in magna dui, in faucibus leo. Nulla nec magna sit amet velit hendrerit semper in in sapien. Nam ullamcorper risus at purus luctus semper nec adipiscing sapien.

\end{multicols}

\end{document}
